\newcommand{\Grid}{\ensuremath{\myyellow{g_1^{0}}}}

\section{Computing all Pointproofs}
\label{s:pointproofs:precompute-all-proofs}

Recall from \cref{eq:indiv-proof} that a proof $\pi_i$ for an element $m_i$ of $\vect{m} = [m_1,\dots,m_N]^\top$ is:
\begin{align}
    \pi_i &= g_1^{\sum_{j\in[N]\setminus\{i\}} m_j \alpha^{j + (N+1) - i}}\\
        \label{eq:indiv-proof-prod}
        &= \prod_{j\in[N]\setminus\{i\}} g_1^{m_j\alpha^{j + (N+1) - i}}
\end{align}
We first give an example for computing all proofs when $N = 4$ and then generalize to arbitrary $N$.
Note that, in this case, our four proofs will be:
\begin{align}
    \pi_1 &= {\hskip 2.5em} g_1^{m_2 \alpha^6} g_1^{m_3\alpha^7} g_1^{m_4\alpha^8}\\
    \pi_2 &= g_1^{m_1 \alpha^4} {\hskip 2.5em} g_1^{m_3\alpha^6} g_1^{m_4\alpha^7}\\
    \pi_3 &= g_1^{m_1 \alpha^3} g_1^{m_2\alpha^4} {\hskip 2.5em} g_1^{m_4\alpha^6}\\
    \pi_4 &= g_1^{m_1 \alpha^2} g_1^{m_2\alpha^3} g_1^{m_3\alpha^4} {\hskip 2.5em}
\end{align}
First, observe that each $\pi_i$ is missing $g_1^{\alpha^{N+1}m_i}$, since the element being proved is not actually part of the proof (see \cref{eq:indiv-proof}).
Second, observe that the exponents $\alpha^j$ shift by one to the right as we move down to the next proof.
Third, observe these equations can be rewritten as inner products.
Specifically, if we let:
\begin{align}
    \vect{a}_1 &=[\Grid, g_1^{\alpha^6}, g_1^{\alpha^7}, g_1^{\alpha^8}]\\
    \vect{a}_2 &=[g_1^{\alpha^4}, \Grid, g_1^{\alpha^6}, g_1^{\alpha^7}]\\
    \vect{a}_3 &=[g_1^{\alpha^3}, g_1^{\alpha^4}, \Grid, g_1^{\alpha^6}]\\
    \vect{a}_3 &=[g_1^{\alpha^2}, g_1^{\alpha^3}, g_1^{\alpha^4}, \Grid]
\end{align}
Then, each proof $\pi_i$ is an inner product between $\vect{a_i}$ and $\vect{m}$:
\begin{align}
    \pi_1 &= \vect{a}_1\vect{m}\\
    \pi_2 &= \vect{a}_2\vect{m}\\
    \pi_3 &= \vect{a}_3\vect{m}\\
    \pi_4 &= \vect{a}_4\vect{m}
\end{align}
As a result, the vector of proofs $\vect{\pi}$ can be computed using a matrix product:
\begin{align}
    \begin{bmatrix}
        \pi_1\\
        \pi_2\\
        \pi_3\\
        \pi_4
    \end{bmatrix} &=
    \begin{bmatrix}
        \llongdash & \vect{a_1} & \rlongdash\\
        \llongdash & \vect{a_2} & \rlongdash\\
        \llongdash & \vect{a_3} & \rlongdash\\
        \llongdash & \vect{a_4} & \rlongdash
    \end{bmatrix}
    \begin{bmatrix}
        m_1\\
        m_2\\
        m_3\\
        m_4
    \end{bmatrix} =
    \begin{bmatrix}
        \Grid & g_1^{\alpha^6} & g_1^{\alpha^7} & g_1^{\alpha^8}\\
        g_1^{\alpha^4} & \Grid & g_1^{\alpha^6} & g_1^{\alpha^7}\\
        g_1^{\alpha^3} & g_1^{\alpha^4} & \Grid & g_1^{\alpha^6}\\
        g_1^{\alpha^2} & g_1^{\alpha^3} & g_1^{\alpha^4} &\Grid
    \end{bmatrix}
    \begin{bmatrix}
        m_1\\
        m_2\\
        m_3\\
        m_4
    \end{bmatrix}
\end{align}
In other words, we have $\vect{\pi} = \matr{A} \vect{m}$, where $\matr{A}$ is the matrix whose $i$th row is $\vect{a}_i$ and $\vect{\pi}$ is a column vector whose $i$th entry is the proof $\pi_i$.
\\

\noindent\textit{Key observation:}
The matrix $\matr{A}$ is a Toeplitz matrix, which means multiplying it by $\vect{m}$ can be done in $O(N\log{N})$ time, as explained in \cref{s:toeplitz}.
As a result, all proofs can be computed in $O(N\log{N})$ time, which is much faster than the naive $O(N^2)$.

\parhead{Generalizing to any $N$.}
Recall that the proof $\pi_i$ is just a commitment to the vector, but ``shifted'' by $(N+1)-i$ to the right.
Looking at $m_j$'s coefficients in \cref{eq:indiv-proof-prod}, we can define $\vect{a}_i$ as:
\begin{align}
    \vect{a}_i &= \left[g_1^{1+(N+1)-i}, g_1^{2+(N+1)-i}, \dots, g_1^{(i-1)+(N+1)-i}, \Grid, g_1^{(i+1)+(N+1)-i}, \dots, g_1^{N+(N+1)-i}\right]
\end{align}
Next, the matrix $\matr{A}$ is just the matrix whose $i$th row is $\vect{a}_i$:
\begin{align}
    \matr{A} &= \begin{bmatrix}
        \llongdash & \vect{a_1} & \rlongdash\\
        \llongdash & \vect{a_2} & \rlongdash\\
         & \vdots & \\
        \llongdash & \vect{a_N} & \rlongdash
    \end{bmatrix}
\end{align}
And all proofs $\vect{\pi} = [\pi_1,\dots,\pi_N]^\top$ can be computed as:
\begin{align}
    \vect{\pi} = \matr{A} \vect{m}
\end{align}
Using the techniques from \cref{s:toeplitz:multiply-vec}, this can be done in $O(N\log{N})$ time.
The key step is constructing a circulant matrix $\matr{C}_{2N}$ that contains $\matr{A}$.
First, $\matr{A}$ is defined by its first column, denoted by $\vect{c}_1$, and first row, denoted by $\vect{a}_1$:
\begin{align}
    \vect{c}_1 &= [\Grid, g_1^N, g_1^{N-1},\dots, g_1^2]^\top\\
    \vect{a}_1 &= [\Grid, g_1^{N+2}, g_1^{N+3},\dots, g_1^{2N}]
\end{align}
As explained in \cref{s:toeplitz:multiply-vec}, the circulant matrix $\matr{C}_{2N}$ will have vector representation:
\begin{align}
    \vect{c}_{2N}
    &= [
        \vect{c_1},
        \Grid,
        g_1^{2N},
        \cdots,
        g_1^{N+3},
        g_1^{N+2}
    ]^\top\\
    &= [
        \Grid, g_1^N, g_1^{N-1},\dots, g_1^2,
        \Grid,
        g_1^{2N},
        \cdots,
        g_1^{N+3},
        g_1^{N+2}
    ]^\top\\
\end{align}
Let $\vect{m'} = [\vect{m}, \vect{0}_N ]^\top$ be a size-$2N$ vector that extends $\vect{m}$ with $N$ zeros.
Then, $\vect{\pi}$ can be computed as the first $N$ entries of $\vect{\pi}'$, which will be of size $2N$:
\begin{align}
    \vect{\pi'} = \invdft(\dft(\vect{c}_{2N})\circ\dft(\vect{m'}))
\end{align}