%
% Colors
%
\definecolor{myBlueColor}{HTML}{268BD2}
\definecolor{myYellowColor}{HTML}{B58900}
\definecolor{myGreenColor}{HTML}{859900}
\definecolor{myRedColor}{HTML}{DC322F}
\definecolor{mygray}{gray}{0.5}
\newcommand{\myblue}[1]{\textcolor{myBlueColor}{#1}}
\newcommand{\myred}[1]{\textcolor{myRedColor}{#1}}
\newcommand{\red}[1]{\textcolor{red}{#1}}
\newcommand{\myyellow}[1]{\textcolor{myYellowColor}{#1}}
\newcommand{\mygreen}[1]{\textcolor{myGreenColor}{#1}}
\newcommand{\g}[1]{\textcolor{ForestGreen}{#1}}
\newcommand{\m}[1]{\myyellow{#1}}
\newcommand{\N}[1]{\myred{#1}}

%
% Math notation
%
\newcommand{\bezout}{B\'ezout\xspace}
\newcommand{\Gr}{\ensuremath{\mathbb{G}}\xspace}
\newcommand{\Z}{\ensuremath{\mathbb{Z}}\xspace}
\newcommand{\Zp}{\ensuremath{\Z_p}\xspace}
\newcommand{\lagr}{\ensuremath{\mathcal{L}}\xspace}
\newcommand{\Oh}{\ensuremath{\mathcal{O}}}  % for Big-Oh notation
\newcommand{\primes}{\ensuremath{\mathsf{Primes}}\xspace}

\newcommand{\vect}[1]{\ensuremath{\bm{#1}}}
\newcommand{\matr}[1]{\ensuremath{\mathbf{#1}}}

% For adding "--- a_i ---" lines to a matrix A whose rows are a_i's
% e.g., \begin{bmatrix} \llongdash a_i \rlongdash\end{bmatrix}
\makeatletter
\newcommand{\longdash}[1][2em]{%
    \makebox[#1]{$\m@th\smash-\mkern-7mu\cleaders\hbox{$\mkern-2mu\smash-\mkern-2mu$}\hfill\mkern-7mu\smash-$}}
\makeatother
\newcommand{\omitskip}{\kern-\arraycolsep}
\newcommand{\llongdash}[1][1.5em]{\longdash[#1]\omitskip}
\newcommand{\rlongdash}[1][1.5em]{\omitskip\longdash[#1]}

%
% Crypto things
%
\newcommand{\poly}{\ensuremath{\mathsf{poly}}\xspace}
\newcommand{\negl}{\ensuremath{\mathsf{negl}}\xspace}
\newcommand{\adv}{\ensuremath{\mathcal{A}}\xspace}
\newcommand{\badv}{\ensuremath{\mathcal{B}}\xspace}

%
% Comments and annotations
%
\newcommand{\authnote}[2]{{\footnotesize\textcolor{red}{{\textbf{#1:} }\textcolor{blue}{#2}}}}
\newcommand{\anote}[1]{{\authnote{Alin}{#1}}}
\newcommand{\atodo}[1]{{\authnote{Alin (\textbf{TODO})}{#1}}}

\newcommand{\nop}{\myred{$\times$}}
\newcommand{\yep}{\textcolor{ForestGreen}{\textbf{\checkmark}}}
\newcommand{\idk}{{?}}

%
% Tables
%
\newcolumntype{R}[2]{%
    >{\adjustbox{angle=#1,lap=\width-(#2)}\bgroup}%
    l%
    <{\egroup}%
}
% e.g. \rot{60}{1em}{Title} rotates the title "Title" by 60 degrees and leaves 1em of space in the column
\newcommand*\rot[2]{\multicolumn{1}{R{#1}{#2}}}% no optional argument here, please!

%
% Formatting
%
\newcommand{\eg}{\textit{e.g.,}\xspace}
\newcommand{\api}{\smallskip \hangindent=\parindent \hangafter=1 \noindent}
\newcommand{\parhead}[1]{\medskip\noindent{\bfseries\boldmath\ignorespaces{#1}}}
\newcommand{\alg}[1]{\ensuremath{\boldmath\mathsf{#1}}}
% for setting custom font sizes
\newlength{\mysize}
\newcommand{\myfontsize}[1]{\setlength{\mysize}{#1pt}%
    \fontsize{\mysize}{1.2\mysize}\selectfont}

% For commenting inside algorithm blocks built using figure/tabular, not algorithmic
\newcommand{\comment}[1]{\ifEurocrypt\scriptsize\else\footnotesize\fi$\triangleright$ \textcolor{mygray}{\textit{#1}}}
\newcommand{\commentcont}[1]{\ifEurocrypt\scriptsize\else\footnotesize\fi\phantom{$\triangleright$} \textcolor{mygray}{\textit{#1}}}

%
% VC things
%

\newcommand{\pp}{\ensuremath{\mathsf{PP}}\xspace}
\newcommand{\prk}{\ensuremath{\mathsf{prk}}\xspace}
\newcommand{\vrk}{\ensuremath{\mathsf{vrk}}\xspace}
\newcommand{\upk}{\ensuremath{\mathsf{upk}}\xspace}

\newcommand{\vcsetup}{\ensuremath{\mathsf{VC.Setup}}\xspace}
\newcommand{\vccommit}{\ensuremath{\mathsf{VC.Commit}}\xspace}
\newcommand{\vcopenpos}{\ensuremath{\mathsf{VC.ProvePos}}\xspace}
\newcommand{\vcverifypos}{\ensuremath{\mathsf{VC.VerifyPos}}\xspace}
\newcommand{\vcverifyupk}{\ensuremath{\mathsf{VC.VerifyUPK}}\xspace}
\newcommand{\vcdigupdate}{\ensuremath{\mathsf{VC.UpdateDig}}\xspace}
\newcommand{\vcproofupdate}{\ensuremath{\mathsf{VC.UpdateProof}}\xspace}
\newcommand{\vcopenall}{\ensuremath{\mathsf{VC.ProveAll}}\xspace}

%
% Theorems, defs, lemmas, etc.
%
\ifNotEurocrypt
\theoremstyle{definition}
\newtheorem{definition}{Definition}[section]
\newtheorem{theorem}{Theorem}[section]
\newtheorem{corollary}{Corollary}[theorem]
\newtheorem{lemma}[theorem]{Lemma}
\fi
